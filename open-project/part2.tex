\documentclass[]{article}
\usepackage[T1]{fontenc}
\usepackage{lmodern}
\usepackage{amssymb,amsmath}
\usepackage{ifxetex,ifluatex}
\usepackage{fixltx2e} % provides \textsubscript
% use upquote if available, for straight quotes in verbatim environments
\IfFileExists{upquote.sty}{\usepackage{upquote}}{}
\ifnum 0\ifxetex 1\fi\ifluatex 1\fi=0 % if pdftex
  \usepackage[utf8]{inputenc}
\else % if luatex or xelatex
  \ifxetex
    \usepackage{mathspec}
    \usepackage{xltxtra,xunicode}
  \else
    \usepackage{fontspec}
  \fi
  \defaultfontfeatures{Mapping=tex-text,Scale=MatchLowercase}
  \newcommand{\euro}{€}
\fi
% use microtype if available
\IfFileExists{microtype.sty}{\usepackage{microtype}}{}
\usepackage{color}
\usepackage{fancyvrb}
\usepackage{fancyhdr}
\newcommand{\VerbBar}{|}
\newcommand{\VERB}{\Verb[commandchars=\\\{\}]}
\DefineVerbatimEnvironment{Highlighting}{Verbatim}{commandchars=\\\{\}}
% Add ',fontsize=\small' for more characters per line
\usepackage{framed}
\definecolor{shadecolor}{RGB}{248,248,248}
\newenvironment{Shaded}{\begin{snugshade}}{\end{snugshade}}
\newcommand{\KeywordTok}[1]{\textcolor[rgb]{0.13,0.29,0.53}{\textbf{{#1}}}}
\newcommand{\DataTypeTok}[1]{\textcolor[rgb]{0.13,0.29,0.53}{{#1}}}
\newcommand{\DecValTok}[1]{\textcolor[rgb]{0.00,0.00,0.81}{{#1}}}
\newcommand{\BaseNTok}[1]{\textcolor[rgb]{0.00,0.00,0.81}{{#1}}}
\newcommand{\FloatTok}[1]{\textcolor[rgb]{0.00,0.00,0.81}{{#1}}}
\newcommand{\CharTok}[1]{\textcolor[rgb]{0.31,0.60,0.02}{{#1}}}
\newcommand{\StringTok}[1]{\textcolor[rgb]{0.31,0.60,0.02}{{#1}}}
\newcommand{\CommentTok}[1]{\textcolor[rgb]{0.56,0.35,0.01}{\textit{{#1}}}}
\newcommand{\OtherTok}[1]{\textcolor[rgb]{0.56,0.35,0.01}{{#1}}}
\newcommand{\AlertTok}[1]{\textcolor[rgb]{0.94,0.16,0.16}{{#1}}}
\newcommand{\FunctionTok}[1]{\textcolor[rgb]{0.00,0.00,0.00}{{#1}}}
\newcommand{\RegionMarkerTok}[1]{{#1}}
\newcommand{\ErrorTok}[1]{\textbf{{#1}}}
\newcommand{\NormalTok}[1]{{#1}}
\ifxetex
  \usepackage[setpagesize=false, % page size defined by xetex
              unicode=false, % unicode breaks when used with xetex
              xetex]{hyperref}
\else
  \usepackage[unicode=true]{hyperref}
\fi
\hypersetup{breaklinks=true,
            bookmarks=true,
            pdfauthor={},
            pdftitle={},
            colorlinks=true,
            citecolor=blue,
            urlcolor=blue,
            linkcolor=magenta,
            pdfborder={0 0 0}}
\urlstyle{same}  % don't use monospace font for urls
\setlength{\parindent}{0pt}
\setlength{\parskip}{6pt plus 2pt minus 1pt}
\setlength{\emergencystretch}{3em}  % prevent overfull lines
\setcounter{secnumdepth}{5}

\definecolor{gray}{rgb}{0.4,0.4,0.4}
\author{}
\date{}

\begin{document}

\pagestyle{fancy}
\lhead{\textcolor{gray}{Simon Robillard \& Inari Listenmaa}}
\rhead{\textcolor{gray}{Advanced Functional Programming}}
\lfoot{\textcolor{gray}{}}
\rfoot{\thepage}
\renewcommand{\headrulewidth}{0.5pt} 
\renewcommand{\footrulewidth}{0.5pt} 
\fancyfoot[C]{\footnotesize \textcolor{gray}{}} 

\centerline{ {\Large \bf Assignment 3 -- part II} }
\vspace*{0.2cm}

As described in part I, we have worked on the Grammatical Framework
package. However, instead of modifying the original random generation
module, we have created our own module, using FEAT
(\href{http://dl.acm.org/citation.cfm?id=2364515}{Dureg{\aa}rd et al.,
2012}).

The paper, implemented in the Haskell library
\href{http://hackage.haskell.org/package/testing-feat}{testing-feat},
describes a technique to construct an enumeration of an algebraic
datatype T as a function \texttt{Nat -\textgreater{} T}. The values
produced by the enumeration are useful for testing, either systematic
(e.g.~with SmallCheck) or random (e.g.~with QuickCheck).

Our contribution to GF is the module \texttt{Enumerate}, which
implements (some of) the same functionalities as the module
\texttt{Generate}, but using a principle based on FEAT. The whole GF
source code, modified to include functions in \texttt{Enumerate}, is available on
\href{https://github.com/inariksit/GF}{github.com/inariksit/GF} and can
be installed using the \texttt{Makefile} provided with the source code. 
We include the module \texttt{Enumerate} in the Fire submission.

We have worked on a module called
\href{http://hackage.haskell.org/package/gf-3.6/docs/PGF.html}{PGF}: an
API meant to be used for embedding GF grammars in Haskell programs. 
Our contributions are thus not visible to a user who is writing GF code 
and using the GF shell, but rather to a user who is writing a Haskell 
program to generate and manipulate GF trees.

\section{Introduction to Grammatical
Framework}\label{introduction-to-grammatical-framework}

Grammatical Framework (GF) is a domain-specific programming language
designed to write multilingual grammars. Each program consists of an
abstract syntax, and 0 or more concrete syntaxes. Below is an example of
a ``Hello world'' grammar.

\begin{Shaded}
\begin{Highlighting}[]
\NormalTok{abstract }\DataTypeTok{Hello} \FunctionTok{=} \NormalTok{\{}

\NormalTok{flags startcat}\FunctionTok{=}\DataTypeTok{Greeting} \NormalTok{;}

\NormalTok{cat}
  \DataTypeTok{Greeting} \NormalTok{;}
  \DataTypeTok{Recipient} \NormalTok{;}
\NormalTok{fun}
  \DataTypeTok{Hello}   \FunctionTok{:} \DataTypeTok{Recipient} \OtherTok{->} \DataTypeTok{Greeting} \NormalTok{;}
  \DataTypeTok{World}   \FunctionTok{:} \DataTypeTok{Recipient} \NormalTok{;}
  \DataTypeTok{Friends} \FunctionTok{:} \DataTypeTok{Recipient} \NormalTok{;}
  \DataTypeTok{Dear}    \FunctionTok{:} \DataTypeTok{Recipient} \OtherTok{->} \DataTypeTok{Recipient} \NormalTok{;}
\NormalTok{\}}
\end{Highlighting}
\end{Shaded}

The grammar has categories \texttt{Greeting} and \texttt{Recipient}, and
functions that build values of these categories. The start category is
\texttt{Greeting}.

In a concrete syntax, we give the categories concrete definitions as
records that consist of strings, tables and parameters; everything that
is needed to represent morphological and syntactical properties of the
categories. In the English example below, we are doing fine with just
string concatenation:

\begin{Shaded}
\begin{Highlighting}[]
\NormalTok{concrete }\DataTypeTok{HelloEng} \KeywordTok{of} \DataTypeTok{Hello} \FunctionTok{=} \NormalTok{\{}

\NormalTok{lincat}
  \DataTypeTok{Greeting}\NormalTok{, }\DataTypeTok{Recipient} \FunctionTok{=} \NormalTok{\{s }\FunctionTok{:} \DataTypeTok{Str}\NormalTok{\} ;}
\NormalTok{lin}
  \DataTypeTok{Hello} \NormalTok{rec }\FunctionTok{=} \NormalTok{\{s }\FunctionTok{=} \StringTok{"hello"} \FunctionTok{++} \NormalTok{rec}\FunctionTok{.}\NormalTok{s\} ;}
  \DataTypeTok{World}     \FunctionTok{=} \NormalTok{\{s }\FunctionTok{=} \StringTok{"world"}\NormalTok{\} ;}
  \DataTypeTok{Friends}   \FunctionTok{=} \NormalTok{\{s }\FunctionTok{=} \StringTok{"friends"}\NormalTok{\} ;}
  \DataTypeTok{Dear}  \NormalTok{rec }\FunctionTok{=} \NormalTok{\{s }\FunctionTok{=} \StringTok{"dear"} \FunctionTok{++} \NormalTok{rec}\FunctionTok{.}\NormalTok{s\} ;}
\NormalTok{\}}
\end{Highlighting}
\end{Shaded}

In Icelandic, greeting agrees with the number of the recipient, so we
need to encode this in our types: \texttt{Greeting} remains as a record
with string, but \texttt{Recipient} includes a field to indicate number.
The function \texttt{Hello} chooses appropriate form depending on its
argument.

\begin{Shaded}
\begin{Highlighting}[]
\NormalTok{concrete }\DataTypeTok{HelloIce} \KeywordTok{of} \DataTypeTok{Hello} \FunctionTok{=} \NormalTok{\{}

\NormalTok{lincat}
  \DataTypeTok{Greeting}  \FunctionTok{=} \NormalTok{\{s }\FunctionTok{:} \DataTypeTok{Str}\NormalTok{\} ;}
  \DataTypeTok{Recipient} \FunctionTok{=} \NormalTok{\{s }\FunctionTok{:} \DataTypeTok{Str} \NormalTok{; n }\FunctionTok{:} \DataTypeTok{Number}\NormalTok{\} ;}
\NormalTok{lin}
  \DataTypeTok{Hello} \NormalTok{rec }\FunctionTok{=} \NormalTok{\{s }\FunctionTok{=} \KeywordTok{case} \NormalTok{rec}\FunctionTok{.}\NormalTok{n }\KeywordTok{of} \NormalTok{\{}
                 \DataTypeTok{Sg} \OtherTok{=>} \StringTok{"sæll"} \FunctionTok{++} \NormalTok{rec}\FunctionTok{.}\NormalTok{s ;}
                     \DataTypeTok{Pl} \OtherTok{=>} \StringTok{"sælir"} \FunctionTok{++} \NormalTok{rec}\FunctionTok{.}\NormalTok{s \} }
              \NormalTok{\} ;}
  \DataTypeTok{World} \FunctionTok{=} \NormalTok{\{s }\FunctionTok{=} \StringTok{"heimur"} \NormalTok{; n }\FunctionTok{=} \DataTypeTok{Sg} \NormalTok{\} ;}
  \DataTypeTok{Friends} \FunctionTok{=} \NormalTok{\{s }\FunctionTok{=} \StringTok{"vinir"} \NormalTok{; n }\FunctionTok{=} \DataTypeTok{Pl} \NormalTok{\} ;}
\NormalTok{param}
  \DataTypeTok{Number} \FunctionTok{=} \DataTypeTok{Sg} \FunctionTok{|} \DataTypeTok{Pl} \NormalTok{;}
\NormalTok{\}}
\end{Highlighting}
\end{Shaded}

We use this grammar as an example in documenting the library interface.

\section{Library interface}\label{library-interface}

The grammars are represented by a data type PGF, which includes the
following information:

\begin{Shaded}
\begin{Highlighting}[]
\KeywordTok{data} \DataTypeTok{PGF} \FunctionTok{=} \DataTypeTok{PGF} \NormalTok{\{}
\OtherTok{  absname   ::} \DataTypeTok{CId} \NormalTok{,}
\OtherTok{  abstract  ::} \DataTypeTok{Abstr} \NormalTok{,}
\OtherTok{  concretes ::} \DataTypeTok{Map.Map} \DataTypeTok{CId} \DataTypeTok{Concr}
  \NormalTok{\}}
\end{Highlighting}
\end{Shaded}

The types \texttt{Abstr} and \texttt{Concr} are not visible in the API;
however, there are a number of functions that take a PGF as an argument
and return relevant information. For example, one can get a list of
categories, functions and languages of a given grammar.

\begin{Shaded}
\begin{Highlighting}[]
\CommentTok{-- List of all languages available in the given grammar}
\OtherTok{languages ::} \DataTypeTok{PGF} \OtherTok{->} \NormalTok{[}\DataTypeTok{Language}\NormalTok{]}

\CommentTok{-- List of all categories defined in the given grammar.}
\OtherTok{categories ::} \DataTypeTok{PGF} \OtherTok{->} \NormalTok{[}\DataTypeTok{CId}\NormalTok{]}

\CommentTok{-- List of all functions defined in the abstract syntax}
\OtherTok{functions ::} \DataTypeTok{PGF} \OtherTok{->} \NormalTok{[}\DataTypeTok{CId}\NormalTok{]}
\end{Highlighting}
\end{Shaded}

Both categories and functions are represented by a type \texttt{CId},
which is just a newtype wrapper for a ByteString. In addition to
\texttt{functions}, the library provides \texttt{functionType} which
takes a PGF and an identifier, and gives the type of that identifier.

\begin{Shaded}
\begin{Highlighting}[]
\CommentTok{-- The type of a given function}
\OtherTok{functionType ::} \DataTypeTok{PGF} \OtherTok{->} \DataTypeTok{CId} \OtherTok{->} \DataTypeTok{Maybe} \DataTypeTok{Type}

\CommentTok{-- List of all functions defined for a given category}
\OtherTok{functionsByCat ::} \DataTypeTok{PGF} \OtherTok{->} \DataTypeTok{CId} \OtherTok{->} \NormalTok{[}\DataTypeTok{CId}\NormalTok{]}
\end{Highlighting}
\end{Shaded}

Applying \texttt{functions} to our Hello grammar, we get the result
\texttt{{[}Dear, Friends, Hello, World{]}}. Applying \texttt{functionType}
to \texttt{World} we get \texttt{DTyp {[}{]} Recipient}, and to
\texttt{Hello} we get \texttt{DTyp {[}Recipient{]} Greeting}, meaning
that the type that we get from applying \texttt{Hello} is a greeting,
and it needs an argument of type \texttt{Recipient}.

\texttt{functionsByCat} takes a category as an argument, and returns a
list of all functions whose return type is that category. For
\texttt{Recipient}, it returns \texttt{{[}Dear,Friends,World{]}}.
\texttt{Dear} takes an argument and the rest two don't, but since all of
them have \texttt{Recipient} as the return type, they are returned in
the same list.

The core functions of a grammar are parsing and linearization. The API
contains also variants where more information is included, such as
probability of the parse.

\begin{Shaded}
\begin{Highlighting}[]
\CommentTok{-- Tries to parse the given string in the specified language}
\CommentTok{-- and to produce abstract syntax expression.}
\OtherTok{parse ::} \DataTypeTok{PGF} \OtherTok{->} \DataTypeTok{Language} \OtherTok{->} \DataTypeTok{Type} \OtherTok{->} \DataTypeTok{String} \OtherTok{->} \NormalTok{[}\DataTypeTok{Tree}\NormalTok{]}

\CommentTok{-- Linearizes given expression as string in the language}
\OtherTok{linearize ::} \DataTypeTok{PGF} \OtherTok{->} \DataTypeTok{Language} \OtherTok{->} \DataTypeTok{Tree} \OtherTok{->} \DataTypeTok{String}
\end{Highlighting}
\end{Shaded}

The generation module provides both exhaustive and random generation of
trees. There are variants such as taking a template (e.g.~only generate
trees of form \texttt{noun verb adjective noun}), or limit generation to
a specified depth.

Our contributions follow the style of the API: we export the functions
\texttt{enumerate()}, which all have a counterpart \texttt{generate*}
function in the API.

\begin{Shaded}
\begin{Highlighting}[]
\CommentTok{-- Generates an exhaustive possibly infinite list of}
\CommentTok{-- abstract syntax expressions.}
\OtherTok{generateAll ::} \DataTypeTok{PGF} \OtherTok{->} \DataTypeTok{Type} \OtherTok{->} \NormalTok{[}\DataTypeTok{Expr}\NormalTok{]}

\CommentTok{-- A variant of 'generateAll' which also takes as argument}
\CommentTok{-- the upper limit of the depth of the generated expression.}
\OtherTok{generateAllDepth ::} \DataTypeTok{PGF} \OtherTok{->} \DataTypeTok{Type} \OtherTok{->} \DataTypeTok{Maybe} \DataTypeTok{Int} \OtherTok{->} \NormalTok{[}\DataTypeTok{Expr}\NormalTok{]}

\CommentTok{-- Generates an infinite list of random  expressions.}
\OtherTok{generateRandom ::} \DataTypeTok{RandomGen} \NormalTok{g }\OtherTok{=>} \NormalTok{g }\OtherTok{->} \DataTypeTok{PGF} \OtherTok{->} \DataTypeTok{Type} \OtherTok{->} \NormalTok{[}\DataTypeTok{Expr}\NormalTok{]}
\end{Highlighting}
\end{Shaded}

In addition to the aspects described above, the library includes
functions for a wide range of tasks, including morphological analysis,
parsing with probabilities and graphical visualization. The full
documentation of the library is in
\href{hackage.haskell.org/package/gf-3.6/docs/PGF.html}{http://hackage.haskell.org/package/gf-3.6/docs/PGF.html}.

\section{Library implementation}\label{library-implementation}

\subsection{Expressions}\label{expressions}

We generate an enumeration of GF abstract syntax trees, represented by a
type \texttt{Expr}. Below is a definition of the type:

\begin{Shaded}
\begin{Highlighting}[]
\KeywordTok{data} \DataTypeTok{Expr} \FunctionTok{=}
   \DataTypeTok{EAbs} \DataTypeTok{BindType} \DataTypeTok{CId} \DataTypeTok{Expr}       \CommentTok{-- ^ lambda abstraction}
 \FunctionTok{|} \DataTypeTok{EApp} \DataTypeTok{Expr} \DataTypeTok{Expr}               \CommentTok{-- ^ application}
 \FunctionTok{|} \DataTypeTok{ELit} \DataTypeTok{Literal}                 \CommentTok{-- ^ literal}
 \FunctionTok{|} \DataTypeTok{EMeta}  \DataTypeTok{MetaId}                \CommentTok{-- ^ meta variable}
 \FunctionTok{|} \DataTypeTok{EFun}   \DataTypeTok{CId}                   \CommentTok{-- ^ function or data constructor}
 \FunctionTok{|} \DataTypeTok{EVar}   \DataTypeTok{Int}                   \CommentTok{-- ^ variable with de Bruijn index}
 \FunctionTok{|} \DataTypeTok{ETyped} \DataTypeTok{Expr} \DataTypeTok{Type}             \CommentTok{-- ^ local type signature}
 \FunctionTok{|} \DataTypeTok{EImplArg} \DataTypeTok{Expr}                \CommentTok{-- ^ implicit argument in expression}
\end{Highlighting}
\end{Shaded}

Most of the definition wasn't relevant for our task. For instance, meta
variables are used in an abstract syntax with dependent types, which is
out of scope for our project. We don't need literals either, since we
want to construct enumerations of the functions defined in the grammar.
The constructors we did use are \texttt{EApp} and \texttt{EFun}; this is
enough to represent all trees in our enumerations.

To demonstrate the difference, here is a sentence parsed in the GF
shell:

\begin{Shaded}
\begin{Highlighting}[]
\DataTypeTok{Duck}\FunctionTok{>} \NormalTok{p }\StringTok{"she made her duck"}
\DataTypeTok{PredVP} \NormalTok{she_Pron (}\DataTypeTok{V2VtoVP} \NormalTok{make_causative_V2V she_Pron duck_V)}
\DataTypeTok{PredVP} \NormalTok{she_Pron (}\DataTypeTok{V2toVP} \NormalTok{make_V2 (}\DataTypeTok{PossNP} \NormalTok{she_Pron duck_N))}
\DataTypeTok{PredVP} \NormalTok{she_Pron (}\DataTypeTok{V3toVP} \NormalTok{make_benefactive_V3 she_Pron (}\DataTypeTok{MassNP} \NormalTok{duck_N))}
\end{Highlighting}
\end{Shaded}

and here in ghci, using PGF library:

\begin{Shaded}
\begin{Highlighting}[]
\FunctionTok{>} \NormalTok{parse duck eng s }\StringTok{"she made her duck"}
\DataTypeTok{EApp} \NormalTok{(}\DataTypeTok{EApp} \NormalTok{(}\DataTypeTok{EFun} \DataTypeTok{PredVP}\NormalTok{)}
           \NormalTok{(}\DataTypeTok{EFun} \NormalTok{she_Pron)) }
     \NormalTok{(}\DataTypeTok{EApp} \NormalTok{(}\DataTypeTok{EApp} \NormalTok{(}\DataTypeTok{EApp} \NormalTok{(}\DataTypeTok{EFun} \DataTypeTok{V2VtoVP}\NormalTok{) }
                       \NormalTok{(}\DataTypeTok{EFun} \NormalTok{make_causative_V2V)) }
                 \NormalTok{(}\DataTypeTok{EFun} \NormalTok{she_Pron)) }
           \NormalTok{(}\DataTypeTok{EFun} \NormalTok{duck_V))}
\FunctionTok{...}
\end{Highlighting}
\end{Shaded}

In the original grammar, we have zero-place functions (e.g.
\texttt{she\_Pron :: Pron}, \texttt{duck\_V :: V}), one-place functions
(e.g. \texttt{MassNP :: N -\textgreater{} NP}), up to three-place
functions (e.g.
\texttt{V3toVP :: V3 -\textgreater{} NP -\textgreater{} NP -\textgreater{} VP}).
The functions in the PGF library are curried: all functions are
\texttt{CId}s that are applied with \texttt{EApp}, no matter what is the
arity of the function in the original GF grammar. This enables us to
create an enumeration for each category, by enumerating all functions
that construct said category. The piece of code below is from the
function \texttt{mkSpaceOfCat}. We can just fold over the list of
arguments, no matter how many they are, since \texttt{EApp} always takes
only two arguments, unlike the functions in the original GF code.

\begin{Shaded}
\begin{Highlighting}[]
\CommentTok{-- given the CId of a function, return its space}
\KeywordTok{case} \NormalTok{functionType abs cons }\KeywordTok{of}
  \DataTypeTok{Nothing} \OtherTok{->} \DataTypeTok{Empty}
  \DataTypeTok{Just} \NormalTok{(}\DataTypeTok{DTyp} \NormalTok{args _ exprs) }\OtherTok{->}
    \KeywordTok{case} \NormalTok{args }\KeywordTok{of}
      \NormalTok{[] }\OtherTok{->} \DataTypeTok{Unit} \NormalTok{(}\DataTypeTok{EFun} \NormalTok{cons)  }\CommentTok{-- Space Expr}
      \NormalTok{_ }\OtherTok{->} \NormalTok{foldl}
           \NormalTok{(liftS2 }\DataTypeTok{EApp}\NormalTok{)  }\CommentTok{-- Space Expr -> Space Expr -> Space Expr}
           \NormalTok{(}\DataTypeTok{Unit} \FunctionTok{$} \DataTypeTok{EFun} \NormalTok{cons) }\CommentTok{-- Space Expr}
           \NormalTok{(map (\textbackslash{}(_,_,}\DataTypeTok{DTyp} \NormalTok{_ cid _) }\OtherTok{->} \NormalTok{mkSpaceOfCat abs cid) args)}
\end{Highlighting}
\end{Shaded}

\subsection{Laziness}\label{laziness}

If a grammar is recursive, enumerating all expressions results in an
infinite list. For example, the Hello world grammar can always add one
\texttt{Dear} more to the recipient. In order to match the original
implementation, we make sure we can make a potentially infinite
enumeration into a list and index it. Every enumeration (called
\texttt{Space} in the code) consists of finite enumerations, one for the
size of the terms. Size \texttt{n} includes all functions with n
constructors: for the Hello grammar, size 1 includes the trees
\texttt{Hello} and \texttt{World}, size 2 has \texttt{Hello World},
\texttt{Hello Friends}, \texttt{Dear World} and \texttt{Dear Friends}.
For flattening the enumeration into a list, we start from size 0, and
add each successive size to the list. The same goes with indexing; we
start searching from the lowest size, and if the desired index isn't
found, we move to the next size.

\section{Analysis of our code}\label{analysis-of-our-code}

Our code is based on a minimal implementation of FEAT ideas written by
Koen Claessen and available at
\href{https://github.com/koengit/feat/blob/master/Feat.hs}{github.com/koengit/feat/Feat.hs}.
In contrast to the actual implementation of FEAT, this one uses a deep
embedding to represent functional enumerations of datatypes, as well as
explicit memoization.

FEAT requires the user to represent the constructors of a datatype as
members of the \texttt{Constructor} in order to produce a functional
enumeration of that type. Alternatively, it is possible to use Template
Haskell to perform compile-time introspection over that datatype. In our
case however, the datatype (the grammar) is already described by a
structure of the type \texttt{PGF}, which is to produce the enumeration.
In fact having two separate deep embeddings, one for the grammar itself,
and one for its enumeration, is a bit superfluous. As future work, it
could be interesting to integrate the enumeration more tightly into the
grammar representation.

Compared to \texttt{Generate}, the module \texttt{Enumerate} improves
the performance of term enumeration. However it is limited to simply
typed terms, whereas \texttt{Generate} can produce dependently typed
terms. It is still unclear whether the same technique could be used to
generate dependently typed terms, but it is an interesting lead for
future work.

\end{document}
